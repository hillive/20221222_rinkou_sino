% twocolumn を使うと2段組になる

%\documentclass[a4j,twocolumn]{jsarticle}        % -> platex
%\documentclass[a4j,twocolumn]{ujarticle}       % -> uplatex
\documentclass[uplatex]{jsarticle}   % -> uplatex + jsarticle

\usepackage{resume} % 他パッケージ,専用コマンド,余白の設定が書かれている

%%%%%%%%%%%%%%%%%%%%%%%%%%%%%%%%%%%%%%%%%%%%%%%%%%%%%%%%%%%%%%%%%%%%%%%%
% ヘッダ: イベント名,日付,所属,タイトル,氏名
%%%%%%%%%%%%%%%%%%%%%%%%%%%%%%%%%%%%%%%%%%%%%%%%%%%%%%%%%%%%%%%%%%%%%%%%

\pagestyle{plain}
\newcommand{\comment}[1]{}
\begin{document}
\twocolumn[
\beginheader{令和n年度 コンピュータサイエンス学部 輪講論文発表}{2022}{12}{22}{井上 研究室}
\title{弦楽合奏団の指揮体験にフィードバック機能を備えたVRコンテンツ}
\author{C0B20032 岡野 真士}
\endheader
]

\vspace{3mm}

 % 本番用ページ番号オフセット

%---------------------------------------------------------------------------
% 本文
%---------------------------------------------------------------------------


\section{はじめに}
 オーケストラを指揮する指揮者は、一生に一度はや ってみたい職業の一つと言われたこともある。

指揮者の主な役割は、楽器を演奏する演奏者のま とめ役となること、また、演奏者の表現力を高めることにある。指揮者から楽団員への指示は、演奏者が持つ指揮 棒とともに指揮者の体全体の動きによる表現で行われる。 右手の動きは、指揮の図形を描き、その図形を描く速度 によってテンポを刻む役割がある。また、指揮に必要 な技術は、曲のテンポを決めること、音の強弱などを指 示すること、指揮者が想像する演奏を演奏者から引き出 し曲全体のムードなどを聴衆に伝えることである。

そこで、本研究では、指揮の基本である曲の拍子や テンポを体験者が思い通りに刻み、また、それができて いるかどうかを確認できるコンテンツを提案する。そのコ ンテンツを用いることで、指揮者の楽しさや難しさを体験 してもらうことを目的とした。
また、指揮の表現に幅をもた せるため、演奏者全体の音量調整とともに、パートごとに音量調整も制御できるようにした。体験者が指揮に没 入して、様々なCG演出を活用できるよう VR 空間を利 用した。

\section{関連研究}
 指揮者を体験するコンテンツについては、これまで多 くの研究がなされてきた。3DCG モデルで表現した楽器 を 2 次元ディスプレイ上に配置し、体験者が指揮者とし て参加する仮想オーケストラ制作に関する報告がある 。この研究では曲の MIDI データの情報をもとに楽器 にアニメーションを加え、LED 付きの指揮棒を用いて体 験者の動きを検出し、曲のテンポへ反映した。て体 験者の動きを検出し、曲のテンポへ反映した。一方、複 数台のスマートフォンにオーケストラの演奏を奏でる楽 器を表示して演奏者と見立ててオーケストラを指揮する 研究も報告された。しかし、この方法では実際に体 験者が見ている映像がスマートフォンに表示されている 楽器であり、オーケストラを目の前にしているという臨場 感を得るのは難しい。

\section{制作したコンテンツ}
 VR 空間上に制作した弦楽合奏団を図 1 に示す。楽 器アバタを使用して弦楽合奏団を表現した。合奏団の パートは、Violin1、Violin2、Viola、Cello、Bass の 5 つと した。コンテンツの楽曲には、4 拍子の曲である、モーツ ァルト作曲のアイネクライネ・ナハトムジークを用いた。指 揮の体験者は両手に持ったコントローラを操作し、全体 の音量とパートごとの音量、曲の再生速度をそれぞれ 独立して制御するようにした。

 再生速度制御に必要なテンポは右手の指揮の動き から推定した。図 2 に 4 拍子の動きとそれを認識する仕 組みを示す。腕が I~IV の順に動いた場合に、4 拍子 が振れていると認識するようにした。4 拍子を表現する腕の動きは、右手に持ったコントロ ーラのY座標の上下移動、X座標の横移動の両方の変 化から判断した。Y座標から検出した 1拍の動作に着目 し、1 拍の間にコントローラが左に移動した場合を 1、右 に移動した場合を 2 とし、それを順に記録していく。現 在の完了時点から過去の動きの 3 つ前まで遡り、それら を並べる。その並び方が、あらかじめ設定した。
 
 
 4 拍子の パターンに 1 つでも該当すれば 4 拍子ができていると判 定した。



%---------------------------------------------------------------------------
% 本文終わり
%---------------------------------------------------------------------------

 % 参考文献
\bibliographystyle{junsrt}
\bibliography{ref}


\end{document}


%-----------------------------------------------------
% テンプレート
%------------------------------------------------------------------------------

%-----------
%% 箇条書き
%-----------
%\begin{itemize}
% \item
%\end{itemize}

%-------------------
%% 番号付き箇条書き
%-------------------
%\begin{enumerate}
% \item
%\end{enumerate}

%-----------
%% 図の表示
%-----------
%\begin{figure}[H]
% \centering
% \includegraphics[clip,width=7cm]{hoge.eps}
% \caption{図タイトル}\label{fig:hoge}
%\end{figure}

%-----------
%% 図の参照
%-----------
%\figref{fig:hoge}

%-----------
%% 表の作成
%-----------
%\begin{table}[H]
% \centering
% \caption{表タイトル}\label{tab:fuga}
% \begin{tabular}{|c|c|c|}\hline
%  hemo & piyo & fuga \\ \hline
%  hemo & piyo & fuga \\ \hline
% \end{tabular}
%\end{table}

%-----------
%% 表の参照
%-----------
%\tabref{tab:fuga}

%-----------
%% 参考文献
%-----------
%\begin{thebibliography}{9}
% \bibitem{piyo} 参考文献
%\end{thebibliography}

%-----------------
%% 参考文献の参照
%-----------------
%\cite{piyo}
